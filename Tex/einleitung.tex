\chapter{Einleitung}
Auflösungsbeschränlung durch Wellenlänge, jedoch bei kürzeren wellenlängen keine als linsen geeigneten optische Elemente.
Freie Elektronen Laser: kurze Wellenlänge und Kohärenz
ermöglichen die Aufnahme von kohärenten streubildern (CDI - coherent diffraction imaging)
Dieses Verfahren erlaubt es Einzelbild Aufnahmen durchzuführen, insbesondee bei biologischen Proben ist es von Interesse, dass neben dem einbringen in den Strahl keine weitere Präparation von nöten ist und somit die Struktur der Probe bis zur Abbildung nicht beeinträchtigt wird. 

Jedoch geht bei der Aufnahme des Streubildes die Phaseninformation verloren, die einen großteil der Informationen trägt.
Hierfür als lösungsansatz die freiflug holographie sowie die iterativen phasenrekonmstruktionen

zum vergleich der methoden es es zweckhaft zunächst simulierte streubilder einzusetzten, sodass das gewünschte Ergebnis bekannt ist.

fdtd und dda für holographie simulationen ungeeignet aufgrund der räumlichen ausdehnung.
näherung aus der wellenoptik

im rahmen dieser arbeit werden zunächst zwei formulierungen der schichtweisen simulation des durchtritts einer wellenfront durch ein dreidimensionales objekt sowie eine aus der born näherung kommende simulation des streubildes durch vergleich mit der mie streutheorie bewertet um die geeignetste methode auszuwählen.

die ansätze zur lösung des phasenproblems, die holographie sowie die iterative phasenrekonstruktion werden anschließend zunächst für 2D Bilder verglichen, um im Anschluss die simulierte austrittswelle hinter einem dreidimensionalen objekt zu rekonstruieren

