\chapter{Einleitung}
Sowohl in der Clusterphysik, wie auch in Bereichen der Biologie und Medizin besitzt die Untersuchung von feinsten Strukturen mit Auflösungen im Nanometerbereich einen immer größeren Stellenwert. Bei der optischen Abbildung besteht eine Auflösungsbeschränkung aufgrund von Beugungserscheinungen in Abhängigkeit von der Wellenlänge (Abbe-Limit). Bei kürzeren Wellenlängen ist jedoch zum einen die Konstruktion von als Linsen geeigneten optischen Elementen erschwert (durch die bei dieser Wellenlänge kaum vom Vakuum verschiedene Brechzahl der Materialien). Freie-Elektronen-Laser erzeugen kohärente Strahlung bei kurzen Wellenlängen mit hoher Brillianz und ermöglichen durch elastische Lichtstreuung die Aufnahme von sogenannten kohärenten Streubildern ohne die Verwendung von optischen Elementen zur Abbildung. Dieses Verfahren wird als \textit{Coherent Diffraction Imaging (CDI)} bezeichnet~\cite{schultz2013chapter7}.

\textit{CDI} erlaubt es, Einzelbildaufnahmen von Proben mit hoher räumlicher und definierter zeitlicher Auflösung durchzuführen. Insbesondere bei biologischen Proben ist es von Interesse, dass im Gegensatz zu anderen Verfahren (wie der Elektronenmikroskopie) durch die Verwendung von Injektoren, die einzelne Partikel der Probe in den Strahl einbringen, kein Träger für die Probe benötigt wird und auch keine Präparation der Probe von Nöten ist. Somit wird deren Struktur bis zur Abbildung nicht beeinträchtigt. Mit \textit{CDI} konnten bereits Aufnahmen von Viruspartikeln sowie Clusterproben angefertigt werden~\cite{seibert2011}. Jedoch geht bei der Aufnahme des Streubildes dessen Phase verloren, die einen großen Teil der Informationen trägt. Dies wird als das Phasenproblem bezeichnet~\cite{shechtman2015}. Hierfür wurde kürzlich als Lösungsansatz die \textit{Freiflug-Holographie} entwickelt, bei der neben der Probe ein zusätzliches Referenzobjekt mit abgebildet wird und aus der Wechselwirkung der gestreuten Wellen der beiden Objekte Informationen gewonnen werden. Weiterhin existieren verschiedene Algorithmen zur sogenannten  iterativen Phasenrekonstruktion.

Zum Vergleich dieser beiden Ansätze ist es zweckhaft, zunächst simulierte Streubilder einzusetzen, sodass die Ergebnisse der Rekonstruktion mit einem bekannten Ausgangsbild verglichen werden können. Es sind in der Literatur verschiedene Ansätze zur Simulation von Streubildern beschrieben, die gebräuchlichsten sind hierbei \textit{FDTD (Finite Difference Time Domain)}, \textit{DDA (Discrete Dipole Approximation)}, die 3D-Fouriertransformation ausgewertet auf der Ewaldkugel (berechnet mittels einer im Frequenzraum nicht äquidistanten schnellen Fouriertransformation) sowie sogenannte \textit{Multislice} Simulationen~\cite{drezek1999,sander2014,hantke2016,hare1994,barke2015}. Die erstgenannten Ansätze sind für die Simulation von Streubildern wie sie für die Evaluation der Holographie benötigt werden nur eingeschränkt geeignet, da bei gewünschter hoher räumlicher Auflösung und Ausdehnung der Objekte der Speicherbedarf das technisch Machbare um Größenordnungen überschreitet. Unter dem Begriff \textit{Multislice} hingegen werden verschiedene Ansätze subsumiert, die auf Näherungen aus der Wellenoptik aufbauen und die Menge an gleichzeitig zu bearbeitenden Daten durch eine schichtweise Formulierung des Problems reduzieren. 

Im Rahmen dieser Arbeit werden zunächst zwei verschiedene Formulierungen der schichtweisen Simulation des Durchtritts einer Wellenfront durch ein dreidimensionales Objekt sowie eine aus der Born-Näherung abgeleitete Simulation des Streubildes, die alle drei unter dem Begriff \textit{Multislice} beschrieben werden, betrachtet. Sie werden durch einen Vergleich mit der Mie-Streutheorie bewertet um die geeignetste Methode für die Simulation ausgedehnter Strukturen auszuwählen. Die beiden Ansätze zur Lösung des Phasenproblems, die Holographie sowie die iterative Phasenrekonstruktion werden anschließend zunächst für 2D Bilder verglichen, um im Anschluss die simulierte Austrittswelle hinter einer dreidimensionalen Nanostruktur aus dem simulierten Streubild zu rekonstruieren.

