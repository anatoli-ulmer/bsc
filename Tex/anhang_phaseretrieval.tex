\chapter{Algorithmen zur Phasenrückgewinnung}
\label{chap:anhang_algos}
Die implementierten iterativen Rekonstruktionsalgorithmen (ER, HIO und RAAR) in ihrer Projektionsschreibweise sind in \Fref{tab:ipr} dargestellt.

\begin{table}[h]
	\centering
	\begin{tabular}{ccc}
		\hline\hline
		Algorithmus 			&Iteration 								&alternative Darstellung\\
		\hline
		ER  					&$P_sP_m\rho_n$ 									&\\
		HIO  					&$\begin{cases}	
									P_m\rho_n  &\text{für } x\in S\\
									\left[\mathbb{1}-\beta P_m\right]\rho_n &\text{für } x\notin S
								   \end{cases}$										
																					&$\left[P_sP_m+P_{\bar{s}}(\mathbb{1}-\beta P_m)\right]\rho_n$\\
		RAAR  					&$\left[\frac{\beta}{2}\left(R_sR_m+\mathbb{1}\right)+\left(1-\beta\right) P_m\right]\rho_n $
																				&$ $\\
		\hline\hline
	\end{tabular}
	\caption[Text für Tabellenverzeichnis]{Tabellenunterschrift.}
	\label{tab:ipr}
\end{table}	
 Die Äquivalenz zwischen den Darstellungen der Iterationen ist nur gegeben, sofern die Realraumbeschränkungen nur den Support beinhalten, d.h. der Realraum-Projektor $P_s$ geschrieben werden kann als 
 \begin{align}
 P_s\rho (x)=\begin{cases}
  \rho (x)  &\text{für } x\in S\\
  \mathbb{0}  &\text{für }x\notin S
 \end{cases}&&
   P_{\bar{s}}\rho (x)=\begin{cases}
   	\mathbb{0} &\text{für } x\in S\\
   	\rho (x)   &\text{für }x\notin S
   	\end{cases}
 \end{align}
  Nur für den ER-Algorithmus ist zusätzlich eine Form implementiert, deren Realraum-Projektor eine reele, positive Lösung erzwingt.

	Die implementierte Form des RAAR Algorithmus' basiert auf folgender Umformung:
	\begin{align*}
%	R_\nu&=(2P_\nu-\mathbb{1})\\
	\frac{\beta}{2}\left(R_sR_m+\mathbb{1}\right)
	&=\frac{\beta}{2}\left((2P_s-\mathbb{1})(2P_m-\mathbb{1})+\mathbb{1}\right)
		=2\beta P_sP_m-\beta P_s-\beta P_m+\beta\mathbb{1}\\
		\frac{\beta}{2}\left(R_sR_m+\mathbb{1}\right)+\left(1-\beta\right) P_m
		&=2\beta P_sP_m+\beta (\mathbb{1}-P_s)+ (1-2\beta)P_m\\
		&=
		\begin{cases}
			P_m &\text{für } r\in S\\
			(1-2\beta)P_m+\beta\mathbb{1}  &\text{für } r\notin S
		\end{cases}\\
	\end{align*}
