\pdfbookmark[0]{Abstract}{abstract}
	\begin{Huge}
		\textbf{Kurzfassung}\vspace{12mm}
	\end{Huge}
	
	Im Bereich der kohärenten Rönten-Streubildgebung besteht die Fragestellung nach geeigneten Bildrekonstruktionsmethoden. Für einen Vergleich zwischen dem Ansatz der Freiflugholographie und der iterativen Phasenrekonstruktion bieten sich synthetische Streubilder an. Die Simulation von Streubildern hinter räumlich ausgedehnten, dreidimensionalen Objekten lässt sich effizient mit Hilfe verschiedener, schichtweise arbeitenden Algorithmen bewerkstelligen. Eine schichtweise Propagation der Wellenfront durch das Streuobjekt (\textit{Multislice Propagation}) liefert hierbei für Kugeln gut mit der Mie-Theorie übereinstimmende Ergebnisse und wird deshalb für die Simulation der synthetischen Streubilder ausgewählt. Der Vergleich der Rekonstruktionsmethoden im zweidimensionalen Fall zeigt die Vorteile der Freiflugholographie mit Wienerentfaltung bezüglich der Anfälligkeit für Rauschen und fehlender Bildinformationen, setzt jedoch im Gegensatz zur iterativen Phasenrekonstruktion eine genau bekannte Referenz voraus. Eine Kombination aus Holographie und iterativer Rekonstruktion verbindet die Vorteile beider Verfahren und liefert auch für die simulierten dreidimensionalen Streubilder gute Ergebnisse.
