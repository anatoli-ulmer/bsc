\pdfbookmark[0]{Abstract}{abstract}
	\begin{Huge}
		\textbf{Kurzfassung}\vspace{12mm}
	\end{Huge}
	
	Im Bereich der kohärenten Röntgenstreuung an nicht-periodischen Nanostrukturen besteht die Fragestellung nach geeigneten Bildrekonstruktionsmethoden. Für einen Vergleich zwischen dem kürzlich vorgestellten Ansatz der Freiflugholographie und der iterativen Phasenrekonstruktion bieten sich synthetische Streubilder an. Die Simulation von Streubildern hinter räumlich ausgedehnten, dreidimensionalen Objekten lässt sich effizient mit Hilfe verschiedener, schichtweise arbeitenden Algorithmen bewerkstelligen. In dieser Arbeit wird gezeigt, dass eine schichtweise Propagation der Wellenfront durch das Streuobjekt (\textit{Multislice Propagation}) hierbei für Kugeln gut mit der Mie-Theorie übereinstimmende Ergebnisse liefert und wird deshalb für die Simulation der synthetischen Streubilder ausgewählt. Der Vergleich der Rekonstruktionsmethoden im zweidimensionalen Fall zeigt die Vorteile der Freiflugholographie mit Wiener-Entfaltung bezüglich der Anfälligkeit für Rauschen und fehlender Bildinformationen, setzt jedoch im Gegensatz zur iterativen Phasenrekonstruktion eine genau bekannte Referenz voraus. Es wird eine Kombination aus Holographie und iterativer Rekonstruktion vorgestellt, die die Vorteile beider Verfahren verbindet und auch für die simulierten dreidimensionalen Streubilder gute Ergebnisse liefert. Der in dieser Arbeit entwickelte Werkzeugkasten aus Simulation, Validierung und Rekonstruktion liefert eine Basis für die weitere Untersuchung von Ansätzen zur Rekonstruktion dreidimensionaler Informationen aus Streubildern.
