\chapter{Theoretischer Hintergrund}
\label{c_theorie}
 
\section{Skalare Beugungstheorie}

Aus den Maxwell-Gleichungen lässt sich unter der Annahme eines sich im Vergleich zur Wellenlänge $\lambda$ nur langsam ändernden Mediums die skalare Helmholzgleichung
\begin{equation}
(\Delta+k^2\eta^2)\phi=0
\end{equation}
mit der Wellenzahl $k$
\begin{equation}
	k=\frac{2\pi}{\lambda}
\end{equation} und der komplexen Brechzahl des Mediums $\eta$ für eine elektromangetische Welle $\phi$ aufstellen. Diese Betrachtung ignoriert den vektoriellen Charakter der Elektrodynamik, insbesondere die Polarisation der Welle wird vernachlässigt. Des weiteren findet eine Separation der Zeitabhängigkeit statt, sodass nur stationäre Probleme betrachtet werden.

Im Bereich der Röntgenbeugung ist aufgrund der geringfügigen Abweichungen der Brechzahlen von Medien zur Vakuumbrechzahl ($\eta=1$) hilfreich $\eta$ als
\begin{alignat}{2}
\label{eq:brechzahl}
	&\eta&&=1-\delta+i\beta=1 + \delta n \\
\label{eq:approxbrechzahl}
	&\eta^2&&\approx 1 + 2\delta n
\end{alignat}
darzustellen. Bei dieser Darstellung quantifiziert $\delta$ die Refraktion, $\beta$ die Absorption \cite[S. 21]{attwood1999}.
Mit dem Ortsvektor $\vec{r}=(x,y,z)^T$ im Realraum und $\vec{q}=(q_x,q_y,q_z)^T$ im Ortsreaum lautet die Fouriertransformation der Wellengleichung mit der Näherung \ref{eq:approxbrechzahl} und unter Beachtung des Faltungstheorems
\begin{equation}
	(-q^2+k^2)\tilde{\phi}=\frac{-2k^2}{(2\pi)^{\sfrac{3}{2}}}(\tilde{\delta \eta} \ast \tilde{\phi})
\end{equation}
Hierbei bezeichnet die Notation $\tilde{f}(q)$ die Fouriertransformierte von $f(r)$. Wird nun die Greensche Funktion  $\tilde{G}$ im Fourierraum als
\begin{equation}
	\tilde{G}=\frac{1}{(2\pi)^{\sfrac{3}{2}}}\frac{2k^2}{q^2-k^2}
\end{equation}
definiert, so ist die inverse Fouriertransformation von $\tilde{G}$ divergent. Durch Hinzufügen von $\epsilon$ (mit $\epsilon\rightarrow 0^+$) im Nenner jedoch lässt sich die retardierte Greensche Funktion 


\begin{align}
G&=\int_{-\infty}^{\infty} \frac{2k^2}{q^2-k^2+0^+} e^{i\vec{q}\vec{r}}\dif^3 \vec{q}\nonumber\\
&\propto\frac{e^{iqr}}{r}
\end{align}
konstruieren \cite{trigg2006}. Für die Wellengleichung gilt somit im Fourier bzw. im Realraum \cite{cowley1995,thibault2007}:
\begin{align}
\tilde{\phi}&=\tilde{G}(\tilde{\delta \eta} \ast \tilde{\phi})\\
\phi&={G}\ast({\delta \eta}  {\phi})
\end{align}

\section{Born-Näherung}
Eine Ansatz für eine Lösung der Wellengleichung XX lässt sich in Form einer Reihe
\begin{equation}
\phi=\phi_0+\phi_1+\phi_2+..
\end{equation}
(mit $\phi_0$ als einlaufende Welle) aufstellen. Ist die Amplitude der gestreuten Welle deutlich geringer als die der einfallenden Welle, so lässt sich in XX $\phi\approx\phi_0$ im mit $G$ zu faltenden Term anwenden. Dies wird als Born Näherung erster Ordnung bezeichnet. Somit gilt für die Lösung der Wellengleichung

\begin{equation}
\phi\approx\phi_0+G\ast({\delta \eta} \phi_0)
\end{equation}
Ist die einfallende Welle eine ebene Welle $\phi_0=e^{ik_0r}$ und wird der Betrachtungsabstand als groß gegenüber XXX angenommen, so lässt mit der Näherung

\begin{equation}
|r-r'|=\sqrt{r^2+r'^2-2rr'}\approx r-\frac{\vec{r'}\vec{r}}{r}=r-\frac{\vec{k}}{k_0}\vec{r'}\approx r
\end{equation}
für die gestreute Welle mit dem Streuvektor $\vec{q}$ ($\vec{k}=\vec{k_0}+\vec{q}$)
\begin{equation}
\int G(\vec{r}-\vec{r}')\delta\eta(\vec{r}') exp^{ik_0r'}  \dif \vec{r}'\approx e^{ik_0r}\int \delta\eta(\vec{r}')e^{-i\vec{q}\vec{r}'}\dif \vec{r}'\propto \mathscr{F}[\delta\eta]
\end{equation}
aufstellen. Somit ist in dieser Näherung die gestreute Welle proportional zur dreidimensional fouriertransformierten Abweichung der Brechzahl von der Vakuumbrechzahl.




\section{Angular-Spectrum Propagation}
Wird eine elektromagnetische Welle $\phi$ durch ihre zweidimensionale Fouriertransformierte $\bar{\phi}$ als
\begin{equation}
\phi(x,y,z)=\frac{1}{\sqrt{2\pi}}\iint_{-\infty}^{\infty}\bar{\phi}(f_x,f_y,z)e^{i(q_xx+q_yy)} \dif q_x \dif q_y
\end{equation}dargestellt und gefordert dass diese im Vakuum die Wellengleichung XXX erfüllt,
so muss $\bar{\phi}$
\begin{equation}
	\frac{\partial ^2}{\partial z^2}\bar{\phi}(q_x,q_y,z)^2+ \left(k^2-\left(q_x+q_y\right)\right)\bar{\phi}(q_x,q_y,z)=0
\end{equation}
erfüllen. Eine Lösung dieser Gleichung ist
\begin{equation}
\bar{\phi}\left(q_x,q_y,\Delta z\right)=\bar{\phi}(q_x,q_y,0)e^{i\Delta z\sqrt{k^2-(q_x+q_y)^2}}\, . 
\end{equation}
Somit lässt sich die Propagation einer Welle im Vakuum als Multiplikation mit einem Exponentialfaktor im Fourierraum beschreiben. Dieses Verfahren wird \textit{Angular-Spectrum} Propagation genannt \cite{goodman2005}. 
Die eingehende Welle bei $z=0$ wird hierbei in ebene Wellen, die sich in verschiedene Richtungen ausbreiten (Angular Spectrum) zerlegt und deren Propagation berechnet. Die ausgehende Welle bei $z=\Delta z$ lässt sich aus $\bar{\phi}\left(q_x,q_y,\Delta z\right)$ durch eine inverse Fouriertransformation bestimmen.



\section{Fresnel- und Fraunhofer-Näherung}
Wird in XXXFormel die Taylor-Näherung 2. Ordnung
\begin{equation}
	\sqrt{k^2-(q_x+q_y)^2}\approx k-\frac{(q_x^2+q_y^2)}{2k}
\end{equation}
durchgeführt und eine zweidimensionale inverse Fouriertransformation angewendet, so erhält man mit dem Faltungstheorem die Fresnel-Näherung 

\begin{align}
\bar{\phi}(x,y, z)&=\bar{\phi}(x,y,0) e^{ik z}e^{\frac{i z}{2k}(q_x^2+q_y^2)}\\
\phi(x,y, z)&=\phi(x,y,0) \ast \left(
\frac{e^{ik z}}{i z \lambda } 
e^{ik\frac{(x^2+y^2)}{2 z}}
\right)
\end{align}
In dieser Näherung wird die Ausbreitung der Welle durch die Faltung dem sogenannten Fresnel-Propagator ausgedrückt. Wird die Faltung explizit ausgeschrieben, kann der Exponent aufgespalten werden


\begin{align}
\phi(x,y, z)&=\frac{e^{ikz}}{iz\lambda}
\int_{-\infty}^\infty 
\phi(\alpha,\beta,0)
e^{\frac{ik}{2z}\left(\left(x-\alpha\right)^2+\left(y-\beta\right)^2\right)}
\dif \alpha \dif \beta \nonumber \\
&=\frac{e^{ikz}}{iz\lambda}e^{\frac{ik}{2z}(x^2+y^2)}
\int_{-\infty}^\infty 
\phi(\alpha,\beta,0)
e^{\frac{ik}{2z}\left(\alpha^2+\beta^2\right)}
e^{\frac{-ik}{ z}\left(x\alpha+y\beta\right)}
\dif \alpha \dif \beta
\end{align}



Verschwindet die eingehende Welle $\phi_(x,y,0)$ außerhalb eines Bereiches $S=[-X,X]\times[-Y,Y]$ und gilt 
\begin{equation}
\forall \alpha,\beta \in S:	z\gg \frac{k}{2}\left(\alpha^2+\beta^2\right) \, , 
\end{equation}
d.h. sind die Ausmaße des Bereiches, auf dem die Welle nicht verschwindet klein gegenüber dem Beobachtungsabstand, so gilt
\begin{equation}
e^{\frac{ik}{2z}\left(\alpha^2+\beta^2\right)}\approx 1
\end{equation}
und Gleichung XXX lässt sich als Fouriertransformation der Eingangswelle ausgewertet bei $f_{x,y}=\tfrac{k}{z}(x,y)$ und multipliziert mit einem in $x,y$-Richtung reinem Phasenterm interpretieren:

\begin{equation}
\phi(x,y)=\frac{e^{ik(z+\frac{x^2+y^2}{2z})}}{iz\lambda}\mathscr{F}\left[\phi_(x,y,0)\right](f_x=\tfrac{kx}{z},f_y=\tfrac{kx}{z}) \, ,
\end{equation}
Diese Näherung wird als Fraunhofer-Näherung bezeichnet.

\begin{comment}

\subsubsection{Subsubsection}

\paragraph{Paragraph}



\section{Abbildungen}
Eine Beispiel-Abbildung ist in Abb. \ref{Abb:BspAbbildung} gezeigt.

\begin{figure}
\centering
\includegraphics[width=0.9\textwidth]{images/SchemaErzeugungNachweis.jpg}
\caption[Abbildungstext im Abbildungsverzeichnis]{Abbildungsunterschrift. Abbildung nach \cite{B-SViel}.}
\label{fig:BspAbbildung}
\end{figure} 



\section{Tabellen}
Eine Vorlage ist in Tabelle \ref{Tab:BspTabelle} gegeben.
 
\begin{table}
\centering
\begin{tabular}{SccrrS}
\hline\hline
{Druck [\si{\milli\bar}]} & Gas &  $T_0$ [K]& $\Gamma^\ast$ & $\langle N \rangle$ & {$R$ [nm]}\\
\hline
1000 & Ar &  300 & 2.325 & 240 & 1,3\\
1000 & Ar &  300 & 2.971 & 426 & 1,6\\
1000 & Xe &  300 & 7.845 & 4.177 & 3,9\\
1000 & Xe &  300 & 10.024 & 6.337 & 4,4\\
\hline
5000 & Ar &  300 & 11.625 & 8.274 & 4,2\\
5000 & Ar &  300 & 14.854 & 12.862 & 4,9\\
5000 & Xe &  300 & 39.226 & 73.863 & 10,1\\
5000 & Xe &  300 & 50.121 & 114.826 & 11,7\\
\hline
10000 & Ar & 300 & 23.250 & 28.812 & 6,3\\
10000 & Ar & 300 & 29.708 & 44.790 & 7,4\\
10000 & Xe & 300 & 78.452 & 257.207 & 15,3\\
10000 & Xe & 300 & 100.243 & 399.848 & 17,7\\
\hline
8000 & Xe &  220 & 127.589 & 617.258 & 20,4\\
8000 & Xe &  220 & 163.029 & 959.574 & 23,7\\
\hline\hline
\end{tabular}
\caption[Text für Tabellenverzeichnis]{Tabellenunterschrift.}
\label{Tab:BspTabelle}
\end{table}

\section{Anführungszeichen}
Anführungszeichen werden global über das Paket \emph{csquotes} gesetzt und wie folgt eingebunden:

\verb+\enquote{Text in Anführungszeichen}+

Ausgabe: \enquote{Text in Anführungszeichen}.



\section{Setzen von Einheiten}
Zum Setzen von Einheiten wird das Package \emph{siunitx} verwendet:

Zahlen: \num{83567}, \num{,23e-16}, \num{3,89+-0,03}

Winkel: \ang{12.3}, \ang{1;2;3}

Wert mit Einheit: $v_{max}=\SI{260}{\m/\s}$, $v_\infty=\SI{173}{\metre\per\second}$, $R = \SI[per-mode=symbol]{8,3144621}{\joule\per\mol\per\kelvin}$

Wert mit Fehler und Einheit: $T_0=\SI{5,234(15)}{\kelvin}$

Celsius: $T_S=\SI{38.6}{\celsius}$ 

Minuten: $t_c=\SI{90}{\min}$

Druck: $p_0=\SI{80}{\bar}$, Hintergrunddruck \SI{1,9e-9}{\torr}


 
\subsection{Einheiten in Tabellen}
Die Verwendung von Einheiten in Tabellen mit dem Paket \emph{siunitx} ist in Tabelle \ref{Tab:BspS} gezeigt.
\begin{table}
\centering
\begin{tabular}{lSSSSS}
\hline\hline
 & {He} &  {Ne} & {Ar} & {Kr} & {Xe}\\
\hline
$\Gamma_{ch}$ [\SI{e16}{\metre\tothe{\num{-2.15}}\kelvin\tothe{\num{-1.2875}}}] & 36,189 & 31,07 & 3,495 & 1,9531 & 231,036\\
$K_{ch}$ [\si{\K\tothe{\num{2.2875}}\per\milli\bar\micro\metre\tothe{\num{0.85}}}] & 34,865 & 18,5 & 16,46 & 2980,982 & 5,554\\
\hline\hline
\end{tabular}
\caption[Beispiel für die \emph{siunitx}-Klasse \enquote{S} in Tabellen]{Mit der Klasse \enquote{S} des \emph{siunitx}-Pakets werden Zahlen in Tabellen am Dezimaltrennzeichen ausgerichtet. Normaler Text muss dann in geschweiften Klammern gesetzt werden.}
\label{Tab:BspS}
\end{table}


\section{Blindtext}
\end{comment}
