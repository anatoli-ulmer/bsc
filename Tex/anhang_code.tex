\chapter{Programmüberblick}
Es folgt ein Überblick über die wichtigsten im Rahmen dieser Arbeit erstellten Matlab-Funktionen. Der vollständige Programmcode ist unter {\url{http://github.com/fzimmermann89/bsc}} abrufbar und liegt der gedruckten Version dieser Arbeit in Form einer CD-ROM bei.
 
%\subsection*{Hilfsfunktionen}
%\begin{description}[style=nextline]
%	\item [\textit{[data]=\textsc{ft2}(data)}]
%		GPU-optimierte Version von fftshift(fft2(fftshift(data))) für grade N
%	\item [\textit{[data]=\textsc{ift2}(data)}]
%		GPU-optimierte Version von fftshift(ifft2(fftshift(data))) für grade N
%\end{description}
\subsection*{Simulation}
\begin{description}
	\item[scatterObjects]Erzeugung von Streuobjekten
	\begin{description}
		\item[\texttt{.cube}] Würfel
		\item[\texttt{.dodecahedron}] Dodekaeder
		\item[\texttt{.icosahedron}] Ikosaeder
		\item[\texttt{.octahedron}] Oktaeder
		\item[\texttt{.tetrahedron}] Tetraeder
		\item[\normalfont Eigenschaften:]$ $
		\begin{description}
			\item[\texttt{beta,delta,radius,positionX/Y/Z,rotationX/Y/Z}]
		\end{description}
		
		\item[\texttt{.sphere}] Kugel
		\item[\normalfont Eigenschaften:]$ $
		\begin{description}
			\item[\texttt{beta,delta,radius,positionX/Y/Z}]
		\end{description}
		
		\item[\texttt{.matrix}] aus Matlab-Matrix laden. Keine weiteren Eigenschaften.
		\item[\normalfont Funktion aller Streuobjekte:]$ $
			\begin{description}
				\item[\texttt{\textit{[getSlice]}=prepareSliceMethod\textit{(this,N,dx,gpu)})}] Gibt eine \texttt{getSlice} Funktion initialisiert für die räumliche Auflösung \textit{dx} und \textit{N}x\textit{N} Pixel des Streuobjektes zurück. Nutzt bei wahrem \textit{gpu} CUDA um auf der Grafikkarte zu rechnen.
				\begin{description}
					\item [\texttt{\textit{[slice]}=getSlice\textit{(this,z)}}] Die bei \textit{z} liegende Schicht des Streuobjektes berechnet durch Ermittlung der die Ebenen schneidenden Kantenlängen und einem optimierten Punkt-in-Polygon Algorithmus \cite{pnpoly}.
				\end{description}
				
				\item[\texttt{\textit{[matrix]}=toMatrix\textit{(objects,N,dx,gpu)}}] Logische \textit{N}x\textit{N}x\textit{N} Matrix der Streuobjekte \textit{objects}.
				\item[\texttt{\textit{[projection]}=projection\textit{(objects,N,dx,gpu)}}] Projektion der Streuobjekte \textit{objects}. Nutzt bei wahrem \textit{gpu} CUDA um auf der Grafikkarte zu rechnen.
			\end{description}
	\end{description}	
\end{description}


\begin{description}[style=nextline]
	\item [\texttt{\textit{[theta,Intensity,S1,S2]}=mie\textit{(lambda,radius,beta,delta,steps)}}]
	Intensität in Mie Streuung unpolarizierten Lichtes an Sphäre mit Radius \textit{radius} (in nm) und Brechzahl n=1-\textit{delta}+i\textit{beta} bei Wellenlänge \textit{lambda} (in nm), ausgewertet in \textit{steps} linearen Schritten des Winkels \textit{theta}.
			
	\item [\texttt{\textit{[scatter]}=mie\_scatter\textit{(lambda,radius,beta,delta,N,dx,offset)}}]
	Streubild nach Mie mit \textit{N}x\textit{N} Punkten und Realraumauflösung \textit{dx}.
			
	\item [\texttt{\textit{[scatter]}=msft\textit{(lambda,objects,N,dx,dz,gpu,sim\_absorption,debug)}}]
	Streubild nach MSFT bei Wellenlänge \textit{lambda} (in nm) der Objekte im cell-array \textit{objects} berechnet mit \textit{N}x\textit{N} Punkten im Abstand \textit{dx} pro Schicht im Abstand \textit{dz}. Ist der optionale Parameter \textit{sim\_absorption} wahr, wird eine grobe Näherung für Absorption durchgeführt. Nutzt bei wahrem \textit{gpu} CUDA um auf der Grafikkarte zu rechnen.
	In \textit{@debug(data,z)} kann eine Funktion übergeben werden, die in jedem Berechnungsschritt mit den aktuellen Daten aufgerufen wird. 
			
	\item [\texttt{\textit{[exitwave]}=multislice\textit{(lambda,objects,N,dx,dz,gpu,debug)}}]
	Austrittswelle nach Multislice Beam Propagation, die Parameter entsprechen den jeweiligen Parametern bei \texttt{msft}
			
	\item [\texttt{\textit{[exitwave]}=thibault\textit{(lambda,objects,N,dx,gpu,debug)}}]
	Austrittswelle nach Thibaults Multislice, die Parameter entsprechen den jeweiligen Parametern bei \texttt{msft}.
			
	\item [\texttt{\textit{[scatter}=exitwave2scatter\textit{(exitwave,lambda,dx)}}]
	Streubild aus Austrittswelle.
	      		
\end{description}


\subsection*{Rekonstruktion}

	
\begin{description}[style=nextline]
		
	\item [\texttt{\textit{[]}=SupportGeneric\textit{()}}]
	Erzeugt aus einem aufgenommen Streubild den Support und das Startbild
		
	\item [\texttt{\textit{[]}=SupportHolo\textit{()}}]
	Erzeugt aus einem mit Holographie aufgenommen Streubild den Support und das Startbild. Nutzt CUDA um auf der Grafikkarte zu rechnen.
		
	\item [\texttt{recon.Plan}] Klasse für iterative Phasenrekonstruktion. 
	\begin{description}[style=nextline]
		\item [\texttt{addStep\textit{(this,step,iterations,\{parameters\})}}]
		Schritt zur Rekonstruktion hinzufügen. Unterstützte Schritte:
		\begin{description}
			\item[\texttt{er, errp, hio, raar}] ER bzw ER mit postiv-realer Einschränkung
			\item[\texttt{hio, raar}] HIO bzw. RAAR. Parameter: $\beta$
			\item[\texttt{sw}] Shrinkwrap Suppotverfeinerung
			\item[\texttt{loosen}] Support erweitern. Parameter: Radius Gaussfilter, Schwellenwert
			\item[\texttt{untwin}] Fienups Ansatz zur Lösung des Zwillingsproblems duchführen
			\item[\texttt{noise}] Rauschen zum Realraum Bild hinzufügen
			\item[\texttt{show}] Aktuellen Stand anzeigen
			\item[\texttt{writeFrame}] Aktuelles Realraumbild in animierte GIF schreiben. Parameter: Dateiname
		\end{description}
		\item[\texttt{\textit{[reconImage,errors]}=run\textit{(this,scatterImage,support,start,mask)}}]	Rekonstruktion des Streubildes \textit{scatterImage} mit \textit{support} durchführen. Ist \textit{start} eine NxNxA Matrix werden A unabhängige Rekonstruktionen durchgeführt. Nur in \textit{mask} wahre Bereiche des Streubildes werden beachtet. Werden zwei Ausgabevariablen genutzt, werden die Realraumfehler mit berechnet. Nutzt CUDA um auf der Grafikkarte zu rechnen.
		\item[\texttt{\textit{[avg,recons,errors]}=runAvg\textit{(this,scatterImage,support,start,mask,useOnlyBest)}}]	Führt je nach Größe der dritten Dimension von \textit{start} eine Anzahl unabhängige Rekonstruktionen durch und nutzt die \textit{useOnlyBest} (optional) am besten übereinstimmenden Rekonstruktionen um einen Mittelwert nach Korrektur von Spiegelung und Translation zu berechnen.
	\end{description}
	\item [\texttt{\textit{[deconvolution]} = wiener\textit{(input,h,noise)}}] Wiener-Entfaltung des komplexen \textit{input} mit \textit{h} unter Berücksichtigung des Rauschspektrums oder skalaren Wertes \textit{noise}. 
\end{description}