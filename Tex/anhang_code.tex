\chapter{Programmüberblick}
Überblick über die wichtigsten erstellten Matlab-Funktionen. Der vollständige Programmcode ist unter XXX abrufbar.
 
%\subsection*{Hilfsfunktionen}
%\begin{description}[style=nextline]
%	\item [\textit{[data]=\textsc{ft2}(data)}]
%		GPU-optimierte Version von fftshift(fft2(fftshift(data))) für grade N
%	\item [\textit{[data]=\textsc{ift2}(data)}]
%		GPU-optimierte Version von fftshift(ifft2(fftshift(data))) für grade N
%\end{description}
\subsection*{Erzeugung von Objekten}
\subsection*{Simulation}


\begin{description}[style=nextline]
	\item [\texttt{\textit{[theta,Intensity,S1,S2]}=mie\textit{(lambda,radius,beta,delta,steps)}}]
		Intensität in Mie Streuung unpolarizierten Lichtes an Sphäre mit Radius \textit{radius} (in nm) und Brechzahl n=1-\textit{delta}+i\textit{beta} bei Wellenlänge \textit{lambda} (in nm), ausgewertet in \textit{steps} linearen Schritten des Winkels \textit{theta})
		
	\item [\texttt{\textit{[scatter]}=mie\_scatter\textit{(lambda,radius,beta,delta,N,dx,offset)}}]
		Streubild nach Mie mit \textit{N}x\textit{N} Punkten und Realraumauflösung \textit{dx}.
		
	\item [\texttt{\textit{[scatter]}=msft\textit{(lambda,objects,N,dx,dz,gpu,sim\_absorption)}}]
		Streubild nach MSFT bei Wellenlänge \textit{lambda} (in nm) der Objekte im cell-array \textit{objects} berechnet mit \textit{N}x\textit{N} Punkten im Abstand \textit{dx} pro Schicht im Abstand \textit{dz}. Ist der optionale Parameter \textit{sim\_absorption} wahr, wird eine grobe Näherung für Absorption durchgeführt.
		
	\item [\texttt{\textit{[exitwave]}=multislice\textit{(lambda,objects,N,dx,dz,gpu)}}]
		Austrittswelle nach Multislice Beam Propagation.
		
	\item [\texttt{\textit{[exitwave]}=thibault\textit{(lambda,objects,N,dx,gpu)}}]
		Austrittswelle nach Thibaults Multislice.
		
\end{description}


\subsection*{Rekonstruktion}

	
\begin{description}[style=nextline]
	
	\item [\texttt{\textit{[]}=SupportGeneric\textit{()}}]
		Erzeugt aus einem aufgenommen Streubild den Support und das Startbild
	
	\item [\texttt{\textit{[]}=SupportHolo\textit{()}}]
		Erzeugt aus einem mit Holographie aufgenommen Streubild den Support und das Startbild
	
	\item [\texttt{recon.Plan}] Klasse für iterative Phasenrekonstruktion. 
	\begin{description}[style=nextline]
		\item [\texttt{addStep\textit{(step,iterations,\{parameters\})}}]
		Schritt zur Rekonstruktion hinzufügen. Unterstützte Schritte:
		\begin{description}
			\item[\texttt{er, errp, hio, raar}] ER bzw ER mit postiv-realer Einschränkung
			\item[\texttt{hio, raar}] HIO bzw. RAAR. Parameter: $\beta$
			\item[\texttt{sw}] Shrinkwrap Suppotverfeinerung
			\item[\texttt{loosen}] Support erweitern. Parameter: Radius Gaussfilter, Schwellenwert
			\item[\texttt{untwin}] Fienups Ansatz zur Lösung des Zwillingsproblems duchführen
			\item[\texttt{noise}] Rauschen zum Realraum Bild hinzufügen
			\item[\texttt{show}] Aktuellen Stand anzeigen
			\item[\texttt{writeFrame}] Aktuelles Realraumbild in animierte GIF schreiben. Parameter: Dateiname
		\end{description}	
	\end{description}
\end{description}

\begin{comment}


Hilfmittel
	ft2
	ift2
	maskfilter
	pad2size
Simulation
	Erzeugun von Objekten
		
	-msft
	-multislice
	-thibault

Rekonstruktion
	-wiener
	-reconstruct
	SW
	holo\_support
	ERiter
	RAARiter
	HIOiter
	\end{comment}