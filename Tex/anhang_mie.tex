\chapter{Mie-Streuung}
\label{chap:anhang_mie}
Für die Intensität beim Streuwinkel $\theta$ gilt in der Mie-Theorie für die Streuung unpolarisierter Strahlung in Abhängigkeit von der Brechzahl $\eta$ der Sphäre (Radius $r$) und dem Parameter $x=r/k$:
\begin{equation}
	I(\theta)\propto\frac{1}{2}\left(\abs{S_1}^2+\abs{S_2}^2\right)
\end{equation} 
mit den Reihen
\begin{align}
	S_1=\sum_j{\frac{2n+1}{n(n+1)}(a_n\pi_n+b_n\tau_n)} &   & S_2=\sum_n{\frac{2n+1}{n(n+1)}(a_n\tau_n+b_n\pi_n)} 
\end{align}
wobei die Reihen bei der numerischen Auswertung nach $N$ Termen abgebrochen werden. Eine hinreichende Konvergenz liegt meist bei $N\approx2+x+4\sqrt[3]{x}$ Termen vor.  $\pi_n$ und $\tau_n$ können mit $\pi_1=1$ und  $\pi_2=3\cos{\theta}$ rekursiv über die Relationen
\begin{align}
	  & \pi_n=\frac{2n-1}{n-1}\cos{\theta}\pi_{n-1}-\frac{n}{n-1}\pi_{n-2} &   & \tau_n=n\cos{\theta}\pi_n-(n+1)\pi_{n-1} \\
\end{align}
und $a_n$,$b_n$ über 
\begin{align}
	a_n=\frac{(D_n/\eta+n/x)\psi_n-\psi_{n-1}}{(D_n/\eta+n/x)\chi_n-\chi_{n-1}} &   &   
	b_n=\frac{(\eta D_n+n/x)\psi_n-\psi_{n-1}}{(\eta D_n+n/x)\chi_n-\chi_{n-1}}
\end{align} definiert werden. $D_n$ lässt sich rekursiv über
\begin{align}
	D_{N+15}=0 &   & D_{n-1}=\frac{n}{\eta*x}-\frac{1}{D_n+n/(\eta x)} 
\end{align}
und $\psi_n$,$\chi_n$ mittels sphärischer Besselfunktionen 1. Art ($j_n$) und 2. Art ($y_n$) als 
\begin{align}
	\psi_n=x j_n &   & \chi_n=x j_n+ixy_n 
\end{align}
berechnen \cite[S. 112f, 95, 127f]{bohren2008}.

