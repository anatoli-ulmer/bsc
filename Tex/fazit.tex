\chapter{Fazit und Ausblick}
In dieser Arbeit gelang es zunächst von der skalaren Beugungstheorie ausgehend die Entstehung von Röntgenstreubildern zu beschreiben. Darauf aufbauend konnte ein effizienter Ansatz für die Simulation von Streubildern ausgedehnter, komplexer Objekte formuliert, implementiert und mit der Mie-Theorie überprüft werden. Es fand jedoch noch kein Vergleich mit den Ergebnissen aus den Ansätzen \textit{DDA} und \textit{FDTD} statt. Dies würde gegenüber der alleinigen Validierung mittels der Mie-Theorie eine Erweiterung bezüglich der Parameter, bei denen der Algorithmus valide Ergebnisse liefert, auf nicht kugelförmige Objekte ermöglichen. 

Im zweiten Teil der Arbeit wurde die Theorie der Rekonstruktion betrachtet und, neben der bekannten Methode der \textit{iterativen Phasenrekonstruktion} und der neueren \textit{In-Flight Holographie} mit Entfaltung, eine Kombination aus diesen beiden Methoden vorgestellt. Die drei betrachteten Rekonstruktionsansätze besitzen unterschiedliche Stärken und Schwächen hinsichtlich Eindeutigkeit, Qualität und Geschwindigkeit. Es konnte gezeigt werden, dass die iterative Phasenrekonstruktion durch Kombination mit der Holographie deutlich erleichtert wird, sodass sie zum einen stabiler, zum anderen deterministisch, gute Ergebnisse liefern kann. An dieser Stelle sind weitere Untersuchungen erfolgversprechend, insbesondere ist ein Vergleich mit in einem geeigneten experimentellen Aufbau gewonnenen Daten empfehlenswert.

Eine offene Frage ist, in wie weit sich dreidimensionale Informationen aus dem Streubild rekonstruieren lassen. Bei den Rekonstruktionsversuchen in dieser Arbeit wurden die 3D-Eigenschaften des Objektes weitestgehend ignoriert und nur die Austrittswelle betrachtet. Bei dieser Betrachtung kommt jedoch der Vorteil der synthetischen Streubilder noch nicht direkt zum Tragen. Eine weitere Betrachtung der Rückpropagation der Austrittswelle zur Refokussierung ist hier nur der erste Schritt -- im weiteren Verlauf ermöglicht der in dieser Arbeit entwickelte Werkzeugkasten aus Simulation und Rekonstruktion eine weitere Untersuchung der Möglichkeiten zur Rekonstruktion der dreidimensionalen Struktur der Probe. 

Sollten weitere Fortschritte im Bereich der 3D-Informationen gemacht werden, so wäre \textit{CDI} eine Methode, die in dieser Hinsicht ein Alleinstellungsmerkmal besitzt: Durch die kurze Zeit, in der eine Aufnahme erfolgt, ließen sich neue Informationen über den Aufbau biologischer Proben gewinnen, kombiniert mit der geringen Beeinflussung der Probe durch die Präparation wäre ein weiterer Einblick in dynamische Prozesse auf Nanoebene möglich. 