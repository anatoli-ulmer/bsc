\chapter{Ausblick}
Es gelang in dieser Arbeit einen Ansatz für die Simulation von Streubilder  ausgedehnter, komplexer Objekte mit der Mie-Theorie zu überprüfen, ss fand jedoch kein Vergleich mit den Ergebnissen aus den Ansätzen \textit{DDA} und \textit{FDTD} statt, die Validierung mittels der Mie-Theorie erfolgte ausschließlich für Kugeln unterschiedlicher Größe und komplexer Brechzahl. Zwar lässt sich hier ein gewisser Schluss auf die Parameter, bei denen der Algorithmus valide Ergebnisse zu liefern scheint stellen, eine weitere Validierung gegenüber diesen Methoden für Objekte, die mit allen drei Ansätzen simuliert werden können würde jedoch die Sicherheit der Aussage steigern.

Die betrachteten Rekonstruktionsansätze besitzen unterschiedliche Stärken und Schwächen hinsichtlich Determinismus, Qualität und Geschwindigkeit. Es konnte gezeigt werden, dass die iterative Phasenrekonstruktion kann durch Nutzung der Holographie erleichtert werden kann und an dieser Stelle weitere Untersuchungen erfolgversprechend sind. Insbesondere wäre diese Beobachtung auch an experimentellen Daten zu überprüfen.

Eine offene Frage ist, in wie weit sich dreidimensionale Informationen aus dem Streubild rekonstruieren lassen. Bei den bisherigen Rekonstruktionsversuchen wurden jedoch die 3D-Eigenschaften des Objektes weitestgehend ignoriert und nur die Austrittswelle betrachtet. Bei dieser Betrachtung kommt jedoch der Vorteil der synthetischen Streubilder noch nicht direkt zum Tragen. Eine weitere Betrachtung der Rückpropagation der Austrittswelle zur Refokussierung ist hier nur der erste Schritt -- im weiteren Verlauf wäre eine Nutzung des hier entwickelten Werkzeugkastens aus Simulation, Validierung und Rekonstruktion zur Untersuchung der Möglichkeiten zur 3D-Rekonstruktion sinnvoll. 

Sollten weitere Fortschritte im Bereich der 3D-Informationen gemacht werden, so wäre \textit{CDI} eine Methode die in dieser hinnsicht ein Alleinstellungsmerkmal besitzt: Durch die kurze Zeit, in der eine Aufnahme erfolgt, ließen sich neue Informationen über den Aufbau biologischer Proben gewinnen, kombiniert mit der geringen Beeinflussung der Probe durch die Präparation wäre ein weiterer Einblich in Prozesse auf Nanoebene möglich. 