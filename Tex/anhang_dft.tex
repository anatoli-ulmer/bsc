\chapter{Fourier Transformation}
Die in dieser Arbeit verwendete Form der Fouriertransformation lautet
\begin{equation}
	\mathscr{F} [f(\vec{x})] (\vec{q})
	=
	\frac{1}{(2\pi)^{\sfrac{n}{2}}}
	\int_{-\infty}^{\infty}
	f(\vec{x})
	e^{-i\vec{q} \cdot \vec{x} } 
	\dif  \vec{x}
\end{equation}
mit der inversen Fouriertransformation
\begin{equation}
	\mathscr{F}^{-1} [\tilde{f}(\vec{q})] (\vec{x})
	=
	\frac{1}{(2\pi)^{\sfrac{n}{2}}}
	\int_{-\infty}^{\infty}
	\tilde{f}(\vec{q})
	e^{i\vec{q} \cdot \vec{x} } 
	\dif  \vec{q} \, .
\end{equation}
Für diese gilt unter anderem: 
\paragraph{komplexe Konjugation}
\begin{align}
	\label{eq:ft_konjugation}
\mathscr{F}^{-1}\left[ \left(\mathscr{F}\left[f(x)\right]\right)^*  \right]	=f(-x)
\end{align}
$\rightarrow$ eine komplexe Konjugation im Fourierraum entspricht einer Spiegelung im Realraum
\paragraph{doppelte Fouriertransformation}
\begin{align}
	\mathscr{F}\left[\mathscr{F}\left[f(x)\right]  \right]	=f(-x)
\end{align}
$\rightarrow$ die zweimalige Anwendung der Fouriertransformation entspricht einer Spiegelung
\paragraph{Faltungstheorem}
\begin{align*}
	\label{eq:ft_faltung}
	\mathscr{F} [f\ast g] & =(2\pi)^{\sfrac{n}{2}}\mathscr{F}[f] \mathscr{F}[g]     \\
	\mathscr{F}[fg]       & =\frac{1}{(2\pi)^{\sfrac{n}{2}}}\mathscr{F}[f]\ast \mathscr{F}[g] \numberthis
\end{align*}
$\rightarrow$ eine Faltung im Realraum entspricht einer Multiplikation im Fourierraum
\paragraph{Korrelationstheorem}
\begin{align}
	\label{eq:ft_korrelation}
	\mathscr{F} [f\otimes g] & =(2\pi)^{\sfrac{n}{2}}\mathscr{F}[f] (\mathscr{F}[g])^*     \\
\end{align}
$\rightarrow$ eine Korrelation mit einer Funktion im Realraum entspricht im Fourierraum einer Multiplikation mit der komplex-konjugierten Funktion

\paragraph{Ableitung}
\begin{align}
\label{eq:ft_ableitung}
	\mathscr{F} [\pd{f(x)}{x}] (q)=-iq	\mathscr{F} [f] (q)   \\
\end{align}
$\rightarrow$ eine Ableitung  im Realraum wird zu einer Multiplikation im Fourierraum.
