\chapter{Fourier Transformation}
Die in dieser Arbeit verwendete Form der Fouriertransformation lautet
\begin{equation}
	\mathscr{F} [f(\vec{x})] (\vec{q})
	 =
	 \frac{1}{(2\pi)^\frac{n}{2}}
	 \int_{-\infty}^{\infty}
	 f(\vec{x})
	 e^{-i\vec{q} \cdot \vec{x} } 
	 \dif  \vec{x}
\end{equation}
mit der inversen Fouriertransformation
\begin{equation}
\mathscr{F}^-1 [\tilde{f}(\vec{q})] (\vec{x})
=
\frac{1}{(2\pi)^\frac{n}{2}}
\int_{-\infty}^{\infty}
\tilde{f}(\vec{q})
e^{i\vec{q} \cdot \vec{x} } 
\dif  \vec{q} \, .
\end{equation}
Für diese gilt unter anderem das 
\paragraph{Faltungstheorem}
\begin{align}
 	\mathscr{F} [f\ast g]&=(2\pi)^\frac{n}{2}\mathscr{F}[f] \mathscr{F}[g]\\
 	\mathscr{F}[fg]&=(2\pi)^\frac{n}{2}\mathscr{F}[f]\ast \mathscr{F}[g]
\end{align}

relevante Eigenschaften der diskreten Fourier Transformation
stichwort: periodisch, aliasing im vgl zur kontinuierlichen, matlab shift nötig
